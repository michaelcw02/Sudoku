\documentclass[conference]{IEEEtran}
\usepackage[utf8]{inputenc}
\usepackage{graphicx}
\usepackage[spanish,english]{babel}

\usepackage[style=ieee]{biblatex}
\usepackage{csquotes}
\addbibresource{referencias.bib}

\begin{document}

%%%%%%%%%%%%%%%%%%%% PORTADA %%%%%%%%%%%%%%%%%%%%%%%%%%%%%%
\begin{titlepage}

\centering
\includegraphics[scale=0.6]{una.png}\\[10mm]

\textbf{
    {\Large Facultad de Ciencias Exactas y Naturales\\[3mm]
    Escuela de Informática y Computación\\[3mm]
    Ingeniería en Sistemas de Información\\[3mm]
    Paradigmas de Programación}\\[15mm]
    {\LARGE Algoritmos para generación y solución eficiente de Sudokus}\\[15mm]
}

\large{Integrantes:}\\[3mm]

\begin{minipage}{0.4\textwidth}
\begin{flushleft} \large
    Andrey Arguedas Espinoza\\
    \texttt{4-0231-0255}\\
    Daniela Armas Sánchez\\
    \texttt{4-0232-0156}\\
\end{flushleft}
\end{minipage}
~
\begin{minipage}{0.4\textwidth}
\begin{flushleft} \large
    Michael Chen Wang\\
    \texttt{1-1629-0538}\\
    Kimberly Olivas Delgado\\
    \texttt{1-1683-0271}\\
\end{flushleft}
\end{minipage}\\[15mm]

\large{Profesor: \\[3mm] Carlos Loría Sáenz}\\[15mm]

\large {II Ciclo, 2017}

\end{titlepage}
\clearpage


\title{Algoritmos para generación y solución eficiente de Sudokus}
\author{Andrey Arguedas, Daniela Armas, Michael Chen, Kimberly Olivas}
\maketitle

%%%%%%%%%%%%%%%%%%%% RESUMEN %%%%%%%%%%%%%%%%%%%%%%%%%%%%%%
\begin{otherlanguage}{spanish}
\begin{abstract}
El desarrollo de una aplicación web que permitiera jugar al Sudoku requirió, aparte de nuevos conocimientos técnicos, la búsqueda y comprensión de las principales características y reglas del juego.

Posteriormente, se debió investigar acerca de algoritmos, técnicas o heurísticas que facilitan la solución o generación de un tablero de Sudoku, de las cuales se seleccionaron el backtrack, naked single y hidden single. 
\end{abstract}
\end{otherlanguage}

%%%%%%%%%%%%%%%%%%%% INTRODUCCIÓN %%%%%%%%%%%%%%%%%%%%%%%%%
\section{Introducción}
El objetivo del presente proyecto es la implementación de la lógica y la arquitectura de un juego de Sudokus en una aplicación web utilizando un stack MEAN (Mongo, Express, Angular y Node), poniendo en práctica la programación funcional y orientada a objetos. 

El Sudoku es un juego matemático inventado a finales de los años 70 basado en un sistema de probabilidades para representar una serie de números sin repetir inventado por Leonhard Euler de Basilea en el siglo XVIII \cite{sudokuWiki}. 

\centering

Consiste en una cuadrícula, por lo general de 9x9, dividida en sub-cuadrículas de 3x3, la cual se debe llenar con números del 1 al 9 en cada columna, cada fila y cada sub-cuadrícula, sin repeticiones ni conflictos en los tres casos. Dichas reglas se debieron tomar en cuenta a la hora de implementar los métodos y algoritmos principales de solución y creación.

Para ello, se debió realizar una investigación y selección de algoritmos eficientes. Se eligió el backtracking como algoritmo principal, y además, las técnicas de hidden single y naked single, donde la primera funciona para los sudokus de nivel medio y la segunda para los sudokus de nivel fácil \cite{hodoku}. Sin embargo, ambas técnicas ayudan a eliminar opciones, que al combinarlas con un algoritmo de solución como lo es el backtrack, se vuelven más eficientes para el programa.


%%%%%%%%%%%%%%%%%%%% DESARROLLO %%%%%%%%%%%%%%%%%%%%%%%%%%%
\section{Descripción de algoritmos usados}
Los algoritmos del proyecto fueron seleccionados según su capacidad de solución o generación de un sudoku. Los mismos se explican a continuación.

\subsection{Algoritmos de solución}
\subsubsection{Backtrack}
El algoritmo de backtracking o “vuelta atrás” es un tipo de búsqueda de fuerza bruta utilizado para encontrar soluciones a problemas que satisfacen restricciones \cite{vueltaAtras}. Dicha técnica hace una búsqueda en profundidad ya que explora completamente una rama buscando la solución antes de pasar a la siguiente rama.

En grandes rasgos, el algoritmo de backtracking usado para resolver el sudoku, visita en orden cada una de las celdas vacías y las va rellenando secuencialmente con un número del 1 al 9 según lo permitan las restricciones. 
Si el número colocado en una celda es válido, el algoritmo se mueve a la siguiente celda y le asigna el primer número de la secuencia. Si el número es inválido, es decir se repite en la fila, columna o sub-cuadrícula de esta celda, el algoritmo le asigna el siguiente número de la secuencia. Si se han probado los 9 dígitos de la secuencia y ninguno cumple con las restricciones, el algoritmo deja esta celda en blanco y retrocede a la celda anterior, aumentando en uno el valor de esta celda. Este proceso se repite hasta que todas las celdas tengan un número válido, es decir hasta que el algoritmo haya resuelto las 81 celdas \cite{sudokuSolving}.

Este algoritmo tiene la ventaja de garantizar una solución para el sudoku sin importar la dificultad que presente, sin embargo posee la desventaja de que el tiempo de resolución puede ser muy lento ya que verifica todas las posibles soluciones.

\subsubsection{Naked Single}
La técnica “único desnudo” o naked single es una de las más simples para la solución de sudokus. 

Consiste en determinar los posibles valores de una celda vacía examinando los valores de las celdas llenas en la fila, columna y sub-cuadrícula de esta celda. Si al final de la revisión, la celda vacía tiene un único posible valor, éste debe ser el valor de la celda \cite{nakedSolving}.

Es importante destacar que cada vez que se encuentre una celda con sólo un posible valor, este se debe eliminar de las posibilidades de todas las celdas de su fila, columna y sub-cuadrícula. Esto quiere decir que, si el valor de una celda queda naked single, puede desencadenar que muchas otras celdas también queden con valores únicos.

Su desventaja es que sólo es capaz de resolver sudokus completos si son de muy baja dificultad. Sin embargo puede ser de utilidad para encontrar una solución parcial del sudoku. De modo que queden algunas celdas resueltas, facilitando así que otra técnica o algoritmo lo solucione en su totalidad.

\subsubsection{Hidden Single}
La técnica de resolución “único oculto” o hidden single también es bastante sencilla pero resulta más eficaz.

Utilizando esta técnica se determinan los posibles valores de todas las celdas vacías en una fila, columna y sub-cuadrícula determinada. Si un posible valor aparece en sólo una celda de una fila y columna que se intersecan o en la sub-cuadrícula que pertenece dicha celda, entonces ese debe ser el valor de la celda \cite{hiddenSolving}. Se dice que es oculto porque éste único valor se encuentra dentro de la celda junto con otros candidatos \cite{individual}.

Al igual que con la técnica naked single, conforme se vayan encontrando los valores correctos de las celdas se van actualizando los posibles valores de todas las celdas vecinas (es decir, de su misma fila, columna o sub-cuadrícula), lo que provocaría que algunas otras celdas puedan quedar hidden single.

Esta técnica es capaz de solucionar por completo sudokus de dificultad media. Sin embargo es mayormente utilizado en conjunto con otras heurísticas para generar un algoritmo de resolución más eficiente.

\subsection{Algoritmos de generación}
Se utiliza como base la idea de backtracking mencionada anteriormente, sin embargo se realizan algunos cambios en la técnica que permitan la generación completa de un sudoku.

En este caso en particular, el algoritmo visita en orden cada una de las celdas vacías y las va rellenando aleatoriamente con un número del 1 al 9. Si el número colocado es válido, el algoritmo elige un número aleatorio de los 9 candidatos y se lo asigna a la siguiente celda. Sin embargo, si el número es inválido, es decir se repite en la fila, columna o sub-cuadrícula de esta celda, el algoritmo le asigna otro número aleatorio que no haya sido utilizado antes para dicha celda. De igual forma si se acaban las 9 posibilidades para la celda, se retrocede a la anterior y se repite el proceso.

%%%%%%%%%%%%%%%%%%%% CONCLUSIONES %%%%%%%%%%%%%%%%%%%%%%%%%
\section{Conclusiones}
%%%%%%%%%%%%%%%%%%%% REFERENCIAS %%%%%%%%%%%%%%%%%%%%%%%%%%
\begin{otherlanguage}{spanish}
\printbibliography %Prints bibliography
\end{otherlanguage}

\end{document}
